% Colors
\def\hDefaultColor{black}

%% Red
\def\hClassColor{Red}
\def\hTypeColor{Red}
\def\hTypeConstructorColor{RedOrange}
\def\hSetColor{Red}

%% Green
\def\hActionColor{PineGreen}
\def\hFunctionColor{BlueGreen}
\def\hFunctorColor{BlueGreen}

%% Blue
\def\hVarColor{NavyBlue}

%% Pink
\def\hConstantColor{WildStrawberry}
\def\hValueConstructorColor{WildStrawberry}

%% Brown
\def\hKeywordColor{Sepia}


% Keywords
\newcommand{\hKeyword}[1]{\textcolor{\hKeywordColor}{\textsf{#1}}}
\newcommand{\hVarKeyword}[1]{\textcolor{\hKeywordColor}{\texttt{#1}}}
\def\hDeclareMathOperator#1{\DeclareMathOperator{#1}{#1Keyword}}

%% case, of, otherwise
\newcommand{\hCaseKeyword}{\hKeyword{case}}
\newcommand{\hOfKeyword}{\hKeyword{of}}
\newcommand{\hOtherwiseKeyword}{\hKeyword{otherwise}}
\DeclareMathOperator{\hCase}{\hCaseKeyword}
\DeclareMathOperator{\hOf}{\hOfKeyword}
\DeclareMathOperator{\hOtherwise}{\hOtherwiseKeyword}

%% class, deriving, instance
\newcommand{\hClassKeyword}{\hKeyword{class}}
\newcommand{\hDerivingKeyword}{\hKeyword{deriving}}
\newcommand{\hInstanceKeyword}{\hKeyword{instance}}
\DeclareMathOperator{\hClass}{\hClassKeyword}
\DeclareMathOperator{\hDeriving}{\hDerivingKeyword}
\DeclareMathOperator{\hInstance}{\hInstanceKeyword}

%% do
\newcommand{\hDoKeyword}{\hKeyword{do}}
\DeclareMathOperator{\hDo}{\hDoKeyword}

%% let, in, where
\newcommand{\hLetKeyword}{\hKeyword{let}}
\newcommand{\hDoLetKeyword}{\hVarKeyword{let}}
\newcommand{\hInKeyword}{\hKeyword{in}}
\newcommand{\hWhereKeyword}{\hKeyword{where}}
\DeclareMathOperator{\hLet}{\hLetKeyword}
\DeclareMathOperator{\hDoLet}{\hDoLetKeyword}
\DeclareMathOperator{\hIn}{\hInKeyword}
\DeclareMathOperator{\hWhere}{\hWhereKeyword}

%% if, then, else
\newcommand{\hIfKeyword}{\hKeyword{if}}
\newcommand{\hThenKeyword}{\hKeyword{then}}
\newcommand{\hElseKeyword}{\hKeyword{else}}
\DeclareMathOperator{\hIf}{\hIfKeyword}
\DeclareMathOperator{\hThen}{\hThenKeyword}
\DeclareMathOperator{\hElse}{\hElseKeyword}

%% infix, infixl, infixr
\newcommand{\hInfixKeyword}{\hKeyword{infix}}
\newcommand{\hInfixlKeyword}{\hKeyword{infixl}}
\newcommand{\hInfixrKeyword}{\hKeyword{infixr}}
\DeclareMathOperator{\hInfix}{\hInfixKeyword}
\DeclareMathOperator{\hInfixl}{\hInfixlKeyword}
\DeclareMathOperator{\hInfixr}{\hInfixrKeyword}

%% type, data, newtype
\newcommand{\hTypeKeyword}{\hKeyword{type}}
\newcommand{\hDataKeyword}{\hKeyword{data}}
\newcommand{\hNewtypeKeyword}{\hKeyword{newtype}}
\DeclareMathOperator{\hType}{\hTypeKeyword}
\DeclareMathOperator{\hData}{\hDataKeyword}
\DeclareMathOperator{\hNewtype}{\hNewtypeKeyword}

%% Lambda
\newcommand{\hLambda}{\backslash}
\newcommand{\hLambdaArrow}{\mapsto}


% Syntax
\newcommand{\hCaseSyntax}[1]{\hCase#1\hOf}
\newcommand{\hDoSyntax}[1]{\hDo\left\{#1\right\}}
\newcommand{\hIfSyntax}[3]{\hIf#1\hThen#2\hElse#3}
\newcommand{\hLambdaSyntax}[2]{\hLambda#1\hLambdaArrow#2}
\newcommand{\hLetSyntax}[2]{\hLet#1\hIn#2}
\newcommand{\hWhereSyntax}[2]{#1\hWhere#2}


% Constant
\newcommand{\hConstant}[1]{\textcolor{\hConstantColor}{#1}}
\newcommand{\hSpecialConstant}[1]{\textcolor{\hConstantColor}{\mathrm{#1}}}

%% Empty, Nothing, Zero
\newcommand{\hEmptyList}{\hConstant{[\,]}}
\newcommand{\hNothing}{\hConstant{\emptyset}}
\newcommand{\hPureNothing}{\hConstant{\varnothing}}
\newcommand{\hZero}{\hConstant{\O}}

%% True, False
\newcommand{\hTrue}{\hSpecialConstant{True}}
\newcommand{\hFalse}{\hSpecialConstant{False}}

%% String
\newcommand{\hStringLiteral}[1]{\text{#1}}


% Variable
\newcommand{\hVar}[1]{\textcolor{\hVarColor}{#1}}
\newcommand{\hSpecialVar}[1]{\textcolor{\hVarColor}{\mathrm{#1}}}


% Anonymous parameter
\newcommand{\hAnonParam}{\diamond}


% Function
\newcommand{\hFunction}[1]{\textcolor{\hFunctionColor}{#1}}
\newcommand{\hFunc}[1]{\hFunction{#1}}
\newcommand{\hSpecialFunction}[1]{\mathop{\textcolor{\hFunctionColor}{\mathrm{#1}}}}

%% id, sin, cos
\newcommand{\hId}{\hSpecialFunction{id}}
\newcommand{\hSin}{\hSpecialFunction{\sin}}
\newcommand{\hCos}{\hSpecialFunction{\cos}}


% Action
\newcommand{\hAction}[1]{\textcolor{\hActionColor}{#1}}
\newcommand{\hSpecialAction}[1]{\textcolor{\hActionColor}{\mathrm{#1}}}


% Functor
\newcommand{\hFunctor}[1]{\textcolor{\hFunctorColor}{\mathfrak{#1}}}

%% Id
\newcommand{\hCatId}{\hFunctor{id}}


% Type name, Type constructor, Type class, Set
\newcommand{\hTypeName}[1]{\textcolor{\hTypeColor}{\textbf{\textit{#1}}}}
\newcommand{\hSpecialTypeName}[1]{\textcolor{\hTypeColor}{\mathbf{#1}}}
\newcommand{\hTypeConstructor}[1]{\textcolor{\hTypeConstructorColor}{\textbf{\textit{#1}}}}
\newcommand{\hSpecialTypeConstructor}[1]{\textcolor{\hTypeConstructorColor}{\mathbf{#1}}}
% \newcommand{\hTypeClass}[1]{\textcolor{\hClassColor}{\textsc{#1}}}
\newcommand{\hSpecialTypeClass}[1]{\textcolor{\hClassColor}{\textsc{#1}}}
\newcommand{\hSet}[1]{\textcolor{\hSetColor}{\mathbb{#1}}}

%% Bool, Int
\newcommand{\hBool}{\hSpecialTypeName{Bool}}
\newcommand{\hInt}{\hSpecialTypeName{Int}}
\newcommand{\hString}{\hSpecialTypeName{String}}

%% List, Maybe
\newcommand{\hList}{\hSpecialTypeConstructor{List}}
\newcommand{\hMaybe}{\hSpecialTypeConstructor{Maybe}}

%% Eq, Ord
\newcommand{\hEq}{\hSpecialTypeClass{Eq}}
\newcommand{\hOrd}{\hSpecialTypeClass{Ord}}


% Type construct syntax
\newcommand{\hTypeConstruct}[2]{{}^{#1}\langle#2\rangle}

%% List, Maybe
\newcommand{\hListConstruct}[1]{[#1]}
\newcommand{\hVarListConstruct}[1]{\hTypeConstruct{\hList}{#1}}
\newcommand{\hMaybeConstruct}[1]{\hTypeConstruct{\hMaybe}{#1}}


% Value constructor
\newcommand{\hValueConstructor}[1]{\textcolor{\hValueConstructorColor}{#1}}
\newcommand{\hSpecialValueConstructor}[1]{\textcolor{\hValueConstructorColor}{\mathrm{#1}}}

%% Left, Right
\newcommand{\hLeftWith}[1]{{}_{\hSpecialValueConstructor{Left}}[\![#1]\!]}
\newcommand{\hRightWith}[1]{{}_{\hSpecialValueConstructor{Right}}[\![#1]\!]}

%% List, Pure, Constant
\newcommand{\hListWith}[1]{[#1]}
\newcommand{\hPureWith}[1]{\llbracket#1\rrbracket}
\newcommand{\hConstantWith}[1]{\llparenthesis#1\rrparenthesis}

%% Tuple
\newcommand{\hUnit}{(\,)}
\newcommand{\hSingleTuppleWith}[1]{\left(#1\right)}
\newcommand{\hPairWith}[2]{\begin{pmatrix}#1\\#2\end{pmatrix}}
\newcommand{\hTripleWith}[3]{\begin{pmatrix}#1\\#2\\#3\end{pmatrix}}

% Decoration
\newcommand{\hListDecor}[1]{\Tilde{#1}}
\newcommand{\hMaybeDecor}[1]{\Dot{#1}}
\newcommand{\hEitherDecor}[1]{\Ddot{#1}}
\newcommand{\hAnyContextDecor}[1]{\Hat{#1}}


% Operator
%% Binary
\DeclareMathOperator{\hAnyOp}{\bigstar}
\DeclareMathOperator{\hApply}{\S}
\DeclareMathOperator{\hFunctorApply}{\odot}
\DeclareMathOperator{\hMap}{\times}
\DeclareMathOperator{\hMaybeMap}{\boxtimes}
\DeclareMathOperator{\hFunctorMap}{\otimes}
\DeclareMathOperator{\hApplicativeMap}{\cilcledast}
\DeclareMathOperator{\hMonadicBind}{\ast}
\DeclareMathOperator{\hCompose}{\centerdot}
\DeclareMathOperator{\hMonadicCompose}{\circ}
\DeclareMathOperator{\hCategoryCompose}{\bullet}
\DeclareMathOperator{\hIsTypeOf}{:\!:}
\DeclareMathOperator{\hIsInSetOf}{\in}
\DeclareMathOperator{\hLetEq}{\triangleeq}
\DeclareMathOperator{\hMemberAt}{!!}
\DeclareMathOperator{\hValueConstructorOr}{\curlyvee}
\DeclareMathOperator{\hAppend}{\oplus}

%% Arrow
\DeclareMathOperator{\hIfSo}{\dashrightarrow}

%% Fold
\DeclareMathOperator*{\hFold}{\bigcup}
\DeclareMathOperator*{\hFoldRight}{\bigsqcup}
